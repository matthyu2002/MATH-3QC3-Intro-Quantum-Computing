\documentclass{article}
\usepackage{graphicx, amsmath, listings}
\usepackage{amssymb, amsthm,enumerate,braket, verbatim}
\usepackage{quantikz}
\usepackage[most]{tcolorbox}
\usepackage[a4paper, left=1in, right=1in, top=1in, bottom=1in]{geometry}
\usepackage{listings}
\usepackage{color}
\renewcommand{\epsilon}{\varepsilon}
\renewcommand{\phi}{\varphi}
\renewcommand{\H}{\mathbb{H}}

\newtcolorbox[auto counter, number within=section]{question}[1][]{
    colback=white, 
    colframe=black,
    fonttitle=\bfseries\sffamily, 
    colbacktitle=white,
    enhanced,
    coltitle=black,
    top=4mm,
    attach boxed title to top center={xshift=-4cm,yshift=-3mm},
    title=QUESTION \thetcbcounter{#1},
    #1
}

\begin{document}
\title{MATH 3QC3 Assignment 3}
\author{Matthew Yu | 400322243 | Yum77}
\date{\today}
\maketitle
In classical computing, an \textit{oracle} for a function $f$ is a "black box" subroutine that, when given input $x$, returns $f(x)$. In quantum computing, we need our operations to be \textit{unitary} (and thus reversible). Therefore, a \textbf{quantum oracle} for $f$ is built as a unitary operator $U_{f}$ acting on two registers: one that holds the input $x$ (in superposition) and one that holds the output. Concretely, if $x$ is an $n$-bit string and $f(x)$ is an $m$-bit string, the oracle acts as:
$$U_{f}:|x\rangle|y\rangle \mapsto |x\rangle|y \boxplus f(x)\rangle$$

where $\boxplus$ denotes bitwise addition modulo 2. This transformation is reversible because from the pair $(x,y\oplus f(x))$, one can recover the original pair $(x,y)$ when applying $U_{f}$ again.

Whenever you see an expression like $|x\rangle|f(x)\rangle$, assume that this "encoding" was done by a \textit{quantum oracle} of the form above.

\section{The Deutsch-Jozsa Algorithm}

Consider the following problem: you are given a function

$$f:\{0,1\}^{n} \rightarrow \{0,1\},$$

which is promised (i.e. you know) to be either:

\begin{itemize}
    \item \textbf{Constant}: \(f(x)\) is the same for all \(x\) (either always 0 or always 1).
    \item \textbf{Balanced}: Exactly half of the inputs yield 0, and the other half yield 1.
\end{itemize}

Classically, in the worst case, one needs multiple queries to \(f\) (can you think of how many on average?) to distinguish the two cases. Quantumly, the Deutsch-Jozsa algorithm can solve this with \textit{just one} query to the quantum oracle \(U_{f}\).

\subsection{Warm up}

Let's consider the single-bit example, where \(f:\{0,1\} \rightarrow \{0,1\}\). Here, there are only two options: \(f\) is constant iff \(f(0) = f(1)\) and otherwise it is balanced. The algorithm acts on two qubits, and

$$U_{f}|x,y\rangle = |x,y \boxplus f(y)\rangle = |x,y \oplus f(x)\rangle$$

\newpage
\begin{question}[title=QUESTION 1a]\\
Matthew Yu - 400322243
\begin{itemize}
    \item Initialize your state to \(|01\rangle\).
    \item Apply Hadamard to both qubits i.e. \((H \otimes H)\).
    \item Apply \(U_{f}\).
    \item Apply Hadamard to the first qubit.
    \item Measure the first qubit.
\end{itemize}
\end{question}

\textbf{Solution 1a:}
\begin{center}
\begin{quantikz}[transparent]
% Qubit 1 (top wire): initialized to |0⟩
\lstick{$\ket{0}$} & \gate{H} & \gate[wires=2]{U_f} & \gate{H} & \meter{} \\
% Qubit 0 (bottom wire): initialized to |1⟩
\lstick{$\ket{1}$} & \gate{H} & & \qw & \qw
\end{quantikz}
\end{center}

\newpage
\begin{question}[title=QUESTION 1b]\\
Matthew Yu - 400322243
Calculate the state after each evolution through the system.
\end{question}

\textbf{Solution 1b:}
\begin{itemize}
    \item We start with the two-qubit state: $$\ket{01}$$ So the first qubit is $\ket{0}$ and the second qubit is $\ket{1}$
    \item We apple the Hadamard to both qubits $H \otimes H$: $$H\ket{0}=\frac{1}{\sqrt{2}}(\ket{0}+\ket{1}), \quad H\ket{1}=\frac{1}{\sqrt{2}}(\ket{0}-\ket{1})$$
    So we get: 
    \begin{align*}
        H(\ket{0}) \otimes H(\ket{1}) &=\frac{1}{\sqrt{2}}(\ket{0}+\ket{1})\otimes\frac{1}{\sqrt{2}}(\ket{0}-\ket{1})\\
        &=\frac{1}{2}(\ket{0}(\ket{0}-\ket{1})+\ket{1}(\ket{0}-\ket{1}))\\
        &=\frac{1}{2}(\ket{00}-\ket{01}+\ket{10}-\ket{11})
    \end{align*}
    \item Apply oracle $U_f$ where $U_f=\ket{x,y\oplus f(x)}$
    \begin{align*}
        U_f (H(\ket{0}) \otimes H(\ket{1}))&=\frac{1}{2}(\ket{0,0\oplus f(0)}-\ket{0,1\oplus f(0)}+\ket{1,0\oplus f(1)}-\ket{1,1 \oplus f(1)})\\
        &=\frac{1}{2}(\ket{0,f(0)}-\ket{0,1\oplus f(0)}+\ket{1, f(1)}-\ket{1,1 \oplus f(1)})
    \end{align*}
    \item Now we apply Hadamard to the first qubit (right most) of each term:
    \begin{align*}
    \text{First Term}: |0, f(0)\rangle
    H|0\rangle |f(0)\rangle &= \frac{1}{\sqrt{2}} (|0\rangle + |1\rangle) |f(0)\rangle.\\
    \text{Second Term}: -|0, 1 \oplus f(0)\rangle
   -H|0\rangle |1 \oplus f(0)\rangle &= -\frac{1}{\sqrt{2}} (|0\rangle + |1\rangle) |1 \oplus f(0)\rangle.\\
    \text{Third Term}: |1, f(1)\rangle
   H|1\rangle |f(1)\rangle &= \frac{1}{\sqrt{2}} (|0\rangle - |1\rangle) |f(1)\rangle.\\
    \text{Fourth Term}: -|1, 1 \oplus f(1)\rangle 
   -H|1\rangle |1 \oplus f(1)\rangle &= -\frac{1}{\sqrt{2}} (|0\rangle - |1\rangle) |1 \oplus f(1)\rangle.
    \end{align*}
    $$\text{Final State} =\frac{1}{2} \cdot \frac{1}{\sqrt{2}} \Big[
    (|0\rangle + |1\rangle) |f(0)\rangle - (|0\rangle + |1\rangle) |1 \oplus f(0)\rangle + (|0\rangle - |1\rangle) |f(1)\rangle - (|0\rangle - |1\rangle) |1 \oplus f(1)\rangle\Big].$$\\
    \item Group the terms based on the first qubit (\( |0\rangle \) and \( |1\rangle \))
    $$\frac{1}{2\sqrt{2}} \Big[
    |0\rangle \big( |f(0)\rangle - |1 \oplus f(0)\rangle + |f(1)\rangle - |1 \oplus f(1)\rangle \big) + |1\rangle \big( |f(0)\rangle - |1 \oplus f(0)\rangle - |f(1)\rangle + |1 \oplus f(1)\rangle \big)\Big].$$
    \item When you measure the first qubit, the outcome depends on whether \( f \) is \textbf{constant} or \textbf{balanced}:\\
    If \( f \) is constant, there are two possibilities:\\
    All outputs are \( 0 \) (\( f(0) = f(1) = 0 \)):
    \begin{align*}
        &=\frac{1}{2\sqrt{2}} \Big[
        |0\rangle \big( |0\rangle - |1 \oplus 0\rangle + |0\rangle - |1 \oplus 0\rangle \big) + |1\rangle \big( |0\rangle - |1 \oplus 0\rangle - |0\rangle + |1 \oplus 0\rangle \big)\Big]\\
        &=\frac{1}{2\sqrt{2}} \Big[
        |0\rangle \big( |0\rangle - |1\rangle + |0\rangle - |1\rangle \big) + |1\rangle \big( |0\rangle - |1\rangle - |0\rangle + |1\rangle \big)\Big]\\
        &=\frac{1}{2\sqrt{2}} \Big[
        |0\rangle \big( 2|0\rangle - 2|1\rangle\big) +\ket{1}(0)\Big]\\
        &=\frac{1}{\sqrt{2}}|0\rangle \big( |0\rangle - 2|1\rangle\big)
    \end{align*}
    We can see that the first qubit is not in a superposition but just as $\ket{0}$. Therefore it will measure 0 with probability 1.\\
    \\
    If all outputs are \( 0 \) (\( f(0) = f(1) = 1 \)):
     \begin{align*}
        &= \frac{1}{2\sqrt{2}} \Big[
        |0\rangle \big( |1\rangle - |0\rangle + |1\rangle - |0\rangle \big) 
        + |1\rangle \big( |1\rangle - |0\rangle - |1\rangle + |0\rangle \big)\Big].\\
        &\frac{1}{2\sqrt{2}} \Big[
        |0\rangle \big( 2|1\rangle - 2|0\rangle \big) + |1\rangle \big( 0 \big)\Big].\\
        &=\frac{1}{\sqrt{2}} |0\rangle \big( |1\rangle - |0\rangle \big).\\
    \end{align*}
     We can see that the first qubit is not in a superposition but just as $\ket{0}$. Therefore it will measure 0 with probability 1.\\
    If $f$ is balanced, we set f(0)=0, f(1)=1 then check f(0)=1, f(1)=0:
    \begin{align*}
        f(0)=0, f(1)=1 &:\frac{1}{2\sqrt{2}} \Big[
        |0\rangle \big( |0\rangle - |1\rangle + |1\rangle - |0\rangle \big)
        + |1\rangle \big( |0\rangle - |1\rangle - |1\rangle + |0\rangle \big)\Big].\\
        &= \frac{1}{\sqrt{2}} |1\rangle \big( |0\rangle - |1\rangle \big).\\
        f(0)=1,f(1)=0&: \frac{1}{2\sqrt{2}} \Big[
        |0\rangle \big( |1\rangle - |0\rangle + |0\rangle - |1\rangle \big)+ |1\rangle \big( |1\rangle - |0\rangle - |0\rangle + |1\rangle \big)\Big].\\
        &=\frac{1}{\sqrt{2}} |1\rangle \big( |1\rangle - |0\rangle \big).
    \end{align*}
    As we can see the first qubit is measured to be $\ket{1}$ with probabiltiy 1. 
\end{itemize}

\newpage
\begin{question}[title=QUESTION 1c]
Explain why the algorithm determines if \(f\) is balanced or not with probability 1.
\end{question}

\textbf{Solution 1c:} As shown in 1b where we measure the first qubit, when half the inputs are 0 and the other half are 1, the $\ket{0}$ part always cancels out leaving just $\ket{1}$. Therefore, we can conclude that when we measure the first qubit, we will always get $\ket{1}$ 100\% of the time if $f$ is balanced.

\newpage
\begin{question}[title=QUESTION 1d]
Suppose now that \(f:\{0,1\}^{n} \rightarrow \{0,1\}\). Assume that \(f\) is either balanced or constant. Explain why, classically, in the worst case we would need at least \(2^{n-1} - 1\) queries to determine if \(f\) is either balanced or constant (hint: think of a particular example of an \(f\)).
\end{question}

\textbf{Solution 1d:} In classical computing, determining whether a function \( f: \{0,1\}^n \rightarrow \{0,1\} \) is constant or balanced requires querying enough inputs to ensure we observe conflicting outputs if \( f \) is balanced. In the worst-case scenario, \( f \) could return the same value (e.g., 0) for as many inputs as possible before revealing its true nature. For a balanced function, exactly half (\(2^{n-1}\)) of the inputs yield 0 and the other half yield 1. If we query \(2^{n-1} - 1\) inputs and all return 0, there are still \(2^{n-1} + 1\) inputs left unchecked. If the next query (the \(2^{n-1}\)-th) also returns 0, we have \(2^{n-1}\) 0s, confirming \( f \) is balanced. If all \(2^{n-1} + 1\) queried inputs return 0, \( f \) must be constant. Therefore, in the worst case, we need to query at least \(2^{n-1} + 1\) inputs to definitively determine whether \( f \) is constant or balanced.

\newpage
\begin{question}[title=QUESTION 1e]
Consider the following algorithm that acts on \(\mathchoice{\mbox{\boldmath$\sqcap$}}{\mbox{\boldmath$\sqcap$}}{\mbox{\boldmath$\sqcap$}}{\mbox{\boldmath$\sqcap$}}^{\otimes n} \otimes \mathchoice{\mbox{\boldmath$\sqcap$}}{\mbox{\boldmath$\sqcap$}}{\mbox{\boldmath$\sqcap$}}{\mbox{\boldmath$\sqcap$}}^{\otimes n}\):
\begin{itemize}
    \item Input the vector \(|0\rangle^{n} \otimes |1\rangle = |0\ldots 0\rangle \otimes | 1\rangle\),
    \item Apply Hadamard to all qubits,
    \item Apply \(U_{f}\) to all qubits,
    \item Apply Hadamard to the first \(n\) qubits,
    \item Measure the first \(n\) qubits.
\end{itemize}
\end{question}

\textbf{Solution 1e:}

\begin{quantikz}
\lstick[wires=4]{$\ket{0}^{\otimes n}$} & \gate{H} & \gate[wires=5]{U_f} & \gate{H} & \meter{} & \qw \\
                                       & \gate{H} &                      & \gate{H} & \meter{} & \qw \\
                                       & \vdots   &                      & \vdots   & \vdots   & \qw \\
                                       & \gate{H} &                      & \gate{H} & \meter{} & \qw \\
\lstick[wires=1]{$\ket{1}$} & \gate{H} &                      & \qw      & \qw      & \qw \\   
\end{quantikz}
\\
This circuit is similar to the one done in part a but uses the input vector $\ket{0}^n$ and $\ket{1}$ instead

\newpage
\begin{question}[title=QUESTION 1f]
Show that the output determines with probability 1 whether \(f\) is balanced or not.
\end{question}

\textbf{Solution 1f:} This problem is set up similar to 1a but instead of just 2 qubits, we have n $\ket{0}$ and $\ket{1}$:
\begin{itemize}
    \item Apply the Hadamard gate \( H \) to each of the first \( n \) qubits and the last qubit:
     $$H^{\otimes n} |0\rangle^{\otimes n} = \frac{1}{\sqrt{2^n}} \sum_{x \in \{0,1\}^n} |x\rangle,$$
     and
     $$H|1\rangle = \frac{|0\rangle - |1\rangle}{\sqrt{2}}.$$
    The combined state becomes:
     $$\frac{1}{\sqrt{2^n}} \sum_{x \in \{0,1\}^n} |x\rangle \otimes \frac{|0\rangle - |1\rangle}{\sqrt{2}}.$$
    \item Apply $U_f$
    $$\frac{1}{\sqrt{2^n}} \sum_{x \in \{0,1\}^n} |x\rangle \otimes \frac{|0 \oplus f(x)\rangle - |1 \oplus f(x)\rangle}{\sqrt{2}}.$$
    Simplifying:
    $$\frac{1}{\sqrt{2^n}} \sum_{x \in \{0,1\}^n} |x\rangle \otimes \frac{|0\rangle - |1\rangle}{\sqrt{2}}.$$
    \item Apply Hadamard to first n qubits:
     $$H^{\otimes n} \left( \frac{1}{\sqrt{2^n}} \sum_{x \in \{0,1\}^n} |x\rangle \right) = \frac{1}{2^n} \sum_{z \in \{0,1\}^n} \left( \sum_{x \in \{0,1\}^n} (-1)^{x \cdot z} \right) |z\rangle.$$
    The state becomes:
     $$\frac{1}{2^n} \sum_{z \in \{0,1\}^n} \left( \sum_{x \in \{0,1\}^n} (-1)^{x \cdot z} \right) |z\rangle \otimes \frac{|0\rangle - |1\rangle}{\sqrt{2}}.$$  
     \item Measure the first $n$ qubits:\\
     The probability of measuring \( z = 0^n \) is:
     \[
     \left| \frac{1}{2^n} \sum_{x \in \{0,1\}^n} (-1)^{x \cdot 0^n} \right|^2 = \left| \frac{1}{2^n} \sum_{x \in \{0,1\}^n} 1 \right|^2 = 1.
     \]
    If \( f \) is constant:\\
     The state remains unchanged, and the measurement result is \( z = 0^n \) with probability \( 1 \).\\
    If \( f \) is balanced:\\
     The interference caused by the Hadamard transform cancels out the amplitude for \( z = 0^n \), and the measurement result is some other \( z \) with probability \( 1 \).

    If the measurement result is \( z = 0^n \), \( f \) is \textbf{constant}.
    If the measurement result is any other \( z \), \( f \) is \textbf{balanced}.

\[
\boxed{\text{The output } z = 0^n \text{ guarantees } f \text{ is constant; any other } z \text{ guarantees } f \text{ is balanced.}}
\]
  
\end{itemize}


\section{Simon's Problem}

Let's consider \(\{0,1\}^{n}\) as the \(n\)-dimensional \(\mathbb{Z}/2\mathbb{Z}\)-vector-space \(V = (\mathbb{Z}/2\mathbb{Z})^{n}\) (recall that \(\mathbb{Z}/2\mathbb{Z}\) is a field, so we can just do linear algebra as usual). Assume that there is some secret (string) vector \(\vec{s} \in V\) (it's a secret from you!). Imagine now that we have a function \(f: V \rightarrow V\), where you are guaranteed that for any \(\vec{x}, \vec{y} \in V\),
\[
f(\vec{x}) = f(\vec{y}) \Leftrightarrow \vec{x} = \vec{y} \text{ or } \vec{x} = \vec{y} +_{V} \vec{s}
\]
(note that addition in \(V\) is just the same as bit-wise addition \(\mod 2\), \(+_{V} \equiv \boxplus\)). Assume we have a quantum oracle that can perform \(f\), i.e.,
\[
U_{f}|x,y\rangle = |x,y \boxplus f(x)\rangle.
\]
\textbf{Question:} Can we discover the secret string \(\vec{s}\)?\\ The following is a non-trivial fact:\\
\textbf{Theorem 0.1.} \textit{Any classical algorithm that solves this problem with probability at least 2/3 for any such \(f\) must evaluate \(f\) on the order of \(O(2^{n/3})\) times.}\\
So solving this problem efficiently on a classical computer requires exponentially more computations as \(n\) grows.

In this exercise, we will see that an algorithm with a quantum sub-algorithm can do much better.

\newpage
\begin{question}[title=QUESTION 2a]
\subsection{Quantum Algorithm}

(a) Show that for any \(\vec{x} \in V\),

\[
H^{\otimes n}|\vec{x}\rangle = \frac{1}{\sqrt{2^{n}}} \sum_{\vec{z} \in V} (-1)^{\vec{x} \cdot \vec{z}} |z\rangle,
\]

where, \(\vec{x} \cdot \vec{z}\) is just the usual dot product over \(\mathbb{Z}/2\mathbb{Z}\), and \(|\vec{v}\rangle\) is just the standard basis vector of \(\mathbb{C}^{\otimes n}\) corresponding to the sequence of 0's and 1's in \(\vec{v}\).
\end{question}

\textbf{Solution 2a:} We know the output for Hadamard on $\ket{0}$ and $\ket{1}$:
$$H\ket{0}=\frac{1}{\sqrt{2}}(\ket{0}+\ket{1}), \quad H\ket{1}=\frac{1}{\sqrt{2}}(\ket{0}-\ket{1})$$ To generalize for $\ket{x_i}$ we observe that the difference between $\ket{0}$ and $\ket{1}$ is a sign change on $\ket{1}$. We can implement a phase factor: $$(-1)^{x_iz_i},\text{ where } z \text{ is the index of the basis state} \ket{z_i} , (z_i\in \{0,1\})$$
For $z_i=0$: $(-1)^{x_i0}=1$ This means the coefficient of $\ket{0}$ is always +1, regardless of $x_i$.\\
For $z_i=1$: $(-1)^{x_i1}=(-1)^{x_i}$ This means the coefficient of $\ket{1}$ is +1 if $x_i=0$ and -1 if $x_i=1$.\\
We can now write Hadamard on $\ket{x_i}$ as: $$H\ket{x_i}=\frac{1}{\sqrt{2}}\sum^1_{z_i=1}(-1)^{x_iz_i}\ket{z_i}$$
Applying $H^{\otimes n}$ to $\ket{\vec{x}}=\ket{x_1x_2...x_n}$ gives: 
$$H^{\otimes n}|\vec{x}\rangle = \bigotimes_{i=1}^n H|x_i\rangle = \frac{1}{\sqrt{2^n}} \sum_{z_1, \dots, z_n \in \{0,1\}} (-1)^{\sum_{i=1}^n x_i z_i} |z_1 z_2 \dots z_n\rangle.$$
\(\sum_{i=1}^n x_i z_i \mod 2 = \vec{x} \cdot \vec{z}\). Thus:
$$\boxed{
H^{\otimes n}|\vec{x}\rangle = \frac{1}{\sqrt{2^n}} \sum_{\vec{z} \in V} (-1)^{\vec{x} \cdot \vec{z}} |\vec{z}\rangle.}$$

\newpage
\begin{question}[title=QUESTION 2b]\\
Matthew Yu - 400322243
(b) Let \(S = \text{span}\{\vec{s}\}\) and \(S^{\perp}\) its orthogonal complement. Use the previous part to show that for \(\vec{x} \in V\), if \(\vec{y} = \vec{x} + \vec{s}\), then

\[
H^{\otimes n}\left(\frac{1}{\sqrt{2}}|\vec{x}\rangle + \frac{1}{\sqrt{2}}|\vec{y}\rangle\right) = \frac{1}{\sqrt{2^{n-1}}} \sum_{\vec{z} \in S^{\perp}} (-1)^{\vec{x}\vec{z}}|\vec{z}\rangle.
\]

\textbf{Important!} What is the dimension of \(S^{\perp}\)? Relatedly, what is the cardinality (size) of \(S^{\perp}\)?
\end{question}

\textbf{Solution 2b:} Dimension and Cardinality of \(S^\perp\):\\
\begin{itemize}
    \item \(S = \text{span}\{\vec{s}\}\) is 1-dimensional (assuming \(\vec{s} \neq \vec{0}\)).
    \item \(S^\perp\) has dimension \(n - 1\) (orthogonal complement in \(n\)-dimensional space).
    \item Cardinality: \(|S^\perp| = 2^{n-1}\).
\end{itemize}
From part 2(a) we have: $$H^{\otimes n}|\vec{x}\rangle = \frac{1}{\sqrt{2^n}} \sum_{\vec{z} \in V} (-1)^{\vec{x} \cdot \vec{z}} |\vec{z}\rangle,$$
Add $s$: $$H^{\otimes n}|\vec{x} + \vec{s}\rangle = \frac{1}{\sqrt{2^n}} \sum_{\vec{z} \in V} (-1)^{(\vec{x} + \vec{s}) \cdot \vec{z}} |\vec{z}\rangle.$$
Adding both: 
\begin{align*}
    H^{\otimes n}\left(\frac{1}{\sqrt{2}}|\vec{x}\rangle + \frac{1}{\sqrt{2}}|\vec{x} + \vec{s}\rangle\right) &= \frac{1}{\sqrt{2^{n+1}}} \sum_{\vec{z} \in V} \left[(-1)^{\vec{x} \cdot \vec{z}} + (-1)^{(\vec{x} + \vec{s}) \cdot \vec{z}}\right] |\vec{z}\rangle.\\
    &=\frac{1}{\sqrt{2^{n+1}}} \sum_{\vec{z} \in V} \left[(-1)^{\vec{x} \cdot \vec{z}} \left[1 + (-1)^{\vec{s} \cdot \vec{z}}\right]\right] |\vec{z}\rangle.
\end{align*}
If \(\vec{s} \cdot \vec{z} = 0\), the coefficient is \(2(-1)^{\vec{x} \cdot \vec{z}}\); otherwise, it cancels to 0. Thus, the sum reduces to vectors \(\vec{z} \in S^\perp\):
$$\boxed{
\frac{1}{\sqrt{2^{n-1}}} \sum_{\vec{z} \in S^\perp} (-1)^{\vec{x} \cdot \vec{z}} |\vec{z}\rangle.}$$

\newpage
\begin{question}[title=QUESTION 2c]
(c) \textbf{IMPORTANT!} Conclude that the result from the previous part is a uniform superposition of kets, each of which encode a vector in \(S^{\perp}\). In other words, by measuring the state, we can determine an element of \(S^{\perp}\) with probability \(2^{1-n}\).
\end{question}

\textbf{Solution 2c:} From part (b), we have the state:
$$H^{\otimes n} \left( \frac{1}{\sqrt{2}} |\vec{x}\rangle + \frac{1}{\sqrt{2}} |\vec{y}\rangle \right) = \frac{1}{\sqrt{2^{n-1}}} \sum_{\vec{z} \in S^\perp} (-1)^{\vec{x} \cdot \vec{z}} |\vec{z}\rangle,$$
Where \(\vec{y} = \vec{x} + \vec{s}\), and \(S^\perp\) is the orthogonal complement of \(S = span\{\vec{s}\}\).\\
The state is a superposition of all vectors \(\vec{z} \in S^\perp\), with each term \(|\vec{z}\rangle\) having an amplitude of \(\frac{1}{\sqrt{2^{n-1}}}\). The phase factor \((-1)^{\vec{x} \cdot \vec{z}}\) does not affect the magnitude of the amplitude, so all terms in the superposition have equal magnitude. This means the state is a \textbf{uniform superposition} over \(S^\perp\).\\
The dimension of \(S^\perp\) is \(n-1\), so the size of \(S^\perp\) is \(|S^\perp| = 2^{n-1}\). Each term \(|\vec{z}\rangle\) in the superposition has probability:
\[
\left| \frac{1}{\sqrt{2^{n-1}}} \right|^2 = \frac{1}{2^{n-1}} = 2^{1-n}.
\]
Since all \(\vec{z} \in S^\perp\) are equally likely, measuring the state yields a uniformly random element of \(S^\perp\), with each outcome occurring with probability \(2^{1-n}\).

\newpage
\begin{question}[title=QUESTION 2d]
(d) Consider the following algorithm:

\begin{itemize}
    \item Initialize with the state \(|0\rangle^{n} \otimes |0\rangle^{n} \in \mathbb{C}^{\otimes n} \otimes \mathbb{C}^{\otimes n}\).
    \item Apply \(H^{\otimes n} \otimes \mathbf{1}\).
    \item Apply \(U_{f}\).
    \item Apply \(H^{\otimes n} \otimes \mathbf{1}\) again.
    \item Measure the first \(n\) qubits.
\end{itemize}

Show that the output encodes a vector \(\vec{z} \in S^{\perp}\).
\end{question}

\textbf{Solution 2d:} 

\begin{center}
\begin{quantikz}[transparent]
% Qubit 1 (top wire): initialized to |0⟩
\lstick{$\ket{0}^n$} & \gate{H} & \gate[wires=2]{U_f} & \gate{H} & \meter{} \\
% Qubit 0 (bottom wire): initialized to |1⟩
\lstick{$\ket{0}^n$} & \gate{H} & & \qw & \qw
\end{quantikz}
\end{center}
\begin{itemize}
    \item Initialize: Start with the state \(|0\rangle^{\otimes n} \otimes |0\rangle^{\otimes n}\).
    \item Apply Hadamard Transform (\(H^{\otimes n}\)): Apply \(H^{\otimes n}\) to the first \(n\) qubits, creating a superposition:
   $$\frac{1}{\sqrt{2^n}} \sum_{x \in V} |x\rangle.$$
    \item Apply Oracle \(U_f\): Apply \(U_f\) to compute \(f(x)\), resulting in:
   $$\frac{1}{\sqrt{2^n}} \sum_{x \in V} |x\rangle \otimes |f(x)\rangle.$$
   \item Apply Hadamard Transform (\(H^{\otimes n}\)) Again: Apply \(H^{\otimes n}\) to the first \(n\) qubits again, transforming the state to:
   $$\frac{1}{2^n} \sum_{x \in V} \sum_{z \in V} (-1)^{x \cdot z} |z\rangle \otimes |f(x)\rangle.$$
   \item Measure the Input Register: Measure the first \(n\) qubits, collapsing the state to \(|z\rangle\) for some \(z \in V\).
\end{itemize}


\newpage
\begin{question}[title=QUESTION 2e]\\
Matthew Yu - 400322243
(e) Conclude that each run of the algorithm yields an element of \(S^{\perp}\) uniformly at random. Suppose now that you have generated \(\{\vec{v}_{1}, \ldots, \vec{v}_{k}\} \subseteq S^{\perp}\) linearly independent vectors. Show that the probability that the next generated vector is not in \(\text{span}\{\vec{v}_{1}, \ldots, \vec{v}_{k}\}\) is \(1 - \frac{2^{k}}{2^{n-1}}\), and so the expected number of runs to find the \(k+1\)-st independent vector \(\vec{v}_{k+1}\) is \(\frac{2^{n-1}}{2^{n-1} - 2^{k}}\) (hint: it's a Bernoulli trial).
\end{question}

\textbf{Solution 2e:} After generating \( k \) linearly independent vectors \( \{\vec{v}_1, \ldots, \vec{v}_k\} \subseteq S^\perp \), the span \( \text{span}\{\vec{v}_1, \ldots, \vec{v}_k\} \) contains \( 2^k \) vectors. Since \( S^\perp \) has \( 2^{n-1} \) vectors in total, the number of vectors not in the span is:
$$2^{n-1} - 2^k.$$
The probability that a new vector is not in the span is:
$$\frac{2^{n-1} - 2^k}{2^{n-1}} = 1 - \frac{2^k}{2^{n-1}}.$$
Each trial is a Bernoulli experiment with success probability \( p = 1 - \frac{2^k}{2^{n-1}} \). The expected number of trials to achieve the first success in a geometric distribution is \( \frac{1}{p} \). Substituting \( p \), the expected number of trials is:
$$\text{Expected trials} = \frac{1}{1 - \frac{2^k}{2^{n-1}}} = \frac{2^{n-1}}{2^{n-1} - 2^k}.$$

\newpage
\begin{question}[title=QUESTION 2f]\\
Matthew Yu - 400322243
(f) \textbf{Optional:} Prove that the expected number of trials to generate \(n-1\) linearly independent vectors is then

\[
\sum_{k=0}^{n-2} \frac{2^{n-1}}{2^{n-1} - 2^{k}} = (n-1) + \sum_{i=1}^{n-1} \frac{1}{2^{i} - 1} \sim O(n)
\]

and so we can generate \(n-1\) linearly independent vectors from \(S^{\perp}\) with on the order of \(n\) trials.
\end{question}

\textbf{Solution 2f:}
\begin{enumerate}
    \item Simplify the Summation\\
    Substitute \( i = n-1 - k \). When \( k \) ranges from \( 0 \) to \( n-2 \), \( i \) ranges from \( 1 \) to \( n-1 \). The summation becomes:
    $$\sum_{i=1}^{n-1} \frac{2^i}{2^i - 1}.$$
    \item Split the Term\\
    Notice that:
    $$\frac{2^i}{2^i - 1} = 1 + \frac{1}{2^i - 1}.$$  
    Thus, the summation splits into:
    $$\sum_{i=1}^{n-1} 1 + \sum_{i=1}^{n-1} \frac{1}{2^i - 1} = (n-1) + \sum_{i=1}^{n-1} \frac{1}{2^i - 1}.$$
    \item Analyze the Second Sum\\
    The series \( \sum_{i=1}^{\infty} \frac{1}{2^i - 1} \) converges to a constant (approximately \( 1.606 \)). Therefore:
    $$\sum_{i=1}^{n-1} \frac{1}{2^i - 1} \leq \sum_{i=1}^{\infty} \frac{1}{2^i - 1} = O(1).$$
\end{enumerate}

\newpage
\begin{question}[title=QUESTION 2g - ]
(g) Suppose we have performed the algorithm \(n\) times, and so (with high probability!) we have a set of linearly independent \(\{\vec{z}_{1}, \ldots, \vec{z}_{n-1}\} \subseteq S^{\perp}\). Show that \(\vec{s} \in V\) is the unique solution to the system of equations

\[
\vec{z}_{1} \cdot \vec{s} = 0
\]
\[
\vdots
\]
\[
\vec{z}_{n-1} \cdot \vec{s} = 0.
\]

(Hint: \(\text{span}(\vec{s}) = \{\vec{0}, \vec{s}\}\)).
\end{question}

\textbf{Solution 2g:}\\
We are given \( n-1 \) linearly independent vectors \( \{\vec{z}_1, \ldots, \vec{z}_{n-1}\} \subseteq S^\perp \), where \( S = \text{span}\{\vec{s}\} \). \text{ We need to show that } \( \vec{s} \) is the unique solution to the system of equations:
\[
\begin{cases}
\vec{z}_1 \cdot \vec{s} = 0, \\
\quad \vdots \\
\vec{z}_{n-1} \cdot \vec{s} = 0.
\end{cases}
\]
The vectors \( \{\vec{z}_1, \ldots, \vec{z}_{n-1}\} \) span \( S^\perp \), which is an \((n-1)\)-dimensional subspace of \( V = (\mathbb{Z}/2\mathbb{Z})^n \). The system \( \vec{z}_i \cdot \vec{s} = 0 \) for \( i = 1, \ldots, n-1 \) defines a homogeneous linear system over \( \mathbb{Z}/2\mathbb{Z} \). Since the \( \vec{z}_i \) are linearly independent, the system has rank \( n-1 \), and the solution space has dimension \( n - (n-1) = 1 \).\\
\\
The solution space consists of all vectors orthogonal to \( S^\perp \). By definition, \( \vec{s} \in S \), and \( S \) is orthogonal to \( S^\perp \), so \( \vec{s} \) satisfies \( \vec{z}_i \cdot \vec{s} = 0 \) for all \( i \). Over \( \mathbb{Z}/2\mathbb{Z} \), the solution space is \( S = \text{span}\{\vec{s}\} \), which contains exactly two vectors: \( \vec{0} \) and \( \vec{s} \). Since \( \vec{0} \) is trivial, the only non-trivial solution is \( \vec{s} \).\\
\\
Therefore, the system \( \vec{z}_i \cdot \vec{s} = 0 \) for \( i = 1, \ldots, n-1 \) has a unique non-trivial solution \( \vec{s} \). \( \vec{s} \) is uniquely determined by these equations.
\[
\boxed{\vec{s} \text{ is the unique solution to the system } \vec{z}_i \cdot \vec{s} = 0 \text{ for } i = 1, \ldots, n-1.}
\]


\end{document}