\documentclass{article}
\usepackage{graphicx, amsmath, listings, float}
\usepackage[most]{tcolorbox}
\newtcolorbox[auto counter]{question}[1][]{colback=white, colframe=black,
fonttitle=\bfseries\sffamily, colbacktitle=white,
enhanced,coltitle=black,top=4mm,
attach boxed title to top center={xshift=-4cm,yshift=-3mm},
title=QUESTION \thetcbcounter,#1}
\usepackage[a4paper, left=1in, right=1in, top=1in, bottom=1in]{geometry}

\begin{document}
\title{MATH 3QC3 - Assignment 4 Solutions}
\author{Matthew Yu | 400322243 | Yum77}
\date{\today}
\maketitle
This set of problems will guide you through a simplified proof that quantum circuits using only certain gates (``Clifford'' gates) can be simulated efficiently on a classical computer. No group theory background is required; all necessary definitions are introduced.

\section*{Overview}
A \textit{Clifford circuit} is built from:
\begin{itemize}
    \item Initial qubits in the state \(0^{\otimes n}\),
    \item A sequence of gates drawn from the set \(\{\text{CNOT}, H, S\}\), where
    \[
    H = \frac{1}{\sqrt{2}}\begin{pmatrix}1 & 1 \\ 1 & -1\end{pmatrix}, \quad S = \begin{pmatrix}1 & 0 \\ 0 & i\end{pmatrix},
    \]
    \item Measurements in the computational (Z) basis (i.e., measuring \(\sigma_z\) on each qubit).
\end{itemize}
We will show that such circuits can be classically simulated in time polynomial in \(n\) (number of qubits) and \(T\) (number of gates).

\section*{Preliminaries: Pauli Operators and Their Binary Representation}
We denote the single-qubit Pauli matrices by \(\sigma_x, \sigma_y, \sigma_z\) (sometimes denoted \(X, Y, Z\)):
\[
\sigma_x = \begin{pmatrix}0 & 1 \\ 1 & 0\end{pmatrix}, \quad \sigma_y = \begin{pmatrix}0 & -i \\ i & 0\end{pmatrix}, \quad \sigma_z = \begin{pmatrix}1 & 0 \\ 0 & -1\end{pmatrix}.
\]


    

\begin{Question}
    \textbf{Problem 1: Representing Single-Qubit Pauli Operators by Two Bits.}
    \begin{enumerate}
        \item[(a)] We will use the notation \((x, z)\) to represent a single-qubit Pauli operator. Define a rule that matches:
        \[
        (0, 0) \mapsto I, \quad (1, 0) \mapsto \sigma_x, \quad (0, 1) \mapsto \sigma_z, \quad (1, 1) \mapsto \sigma_y.
        \]
        Each bit in the pair \((x, z)\) represents the presence (1) or absence (0) of a \(\sigma_x\) or \(\sigma_z\) factor. Why does \((1, 1)\) correspond to \(\sigma_y\) (up to a global phase)?
        
        \item[(b)] Briefly explain why ignoring the global phase does not affect measurement outcomes in quantum mechanics. (You do not need a rigorous proof, just the physical reasoning.)
    \end{enumerate}
\end{Question}

\begin{question}
    \textbf{Problem 2: Extending to \(n\) Qubits.}
    \begin{enumerate}
        \item[(a)] An \(n\)-qubit Pauli operator is a tensor product of single-qubit Pauli operators, e.g., \(\sigma_x \otimes \sigma_z \otimes \cdots \otimes I\). Explain how we can store an \(n\)-qubit Pauli operator by a \(2n\)-bit string:
        \[
        (x_1, z_1) \parallel (x_2, z_2) \parallel \cdots \parallel (x_n, z_n).
        \]
        
        \item[(b)] Give two examples of 3-qubit Pauli operators (e.g., \(\sigma_y \otimes I \otimes \sigma_z\)) and write them in the above bit-string notation.
    \end{enumerate}
\end{question}

\section*{Actions of the Clifford Gates on Pauli Operators}
The key fact is that each Clifford gate \(G\) \textit{conjugates} Pauli operators to Pauli operators (up to a global phase). Explicitly, \(G \sigma G^\dagger\) is again some Pauli operator. In the bit form, this conjugation corresponds to simple \textit{bitwise updates}.

\begin{question}
    \textbf{Problem 3: Single-Qubit Gate Conjugation.}\\
    Consider a single qubit on which you apply \(H\) or \(S\). Suppose this qubit is acted on by a Pauli operator with bits \((x, z)\).
    \begin{enumerate}
        \item[(a)] \textbf{Hadamard gate \(H\).} Show that
        \[
        H \sigma_x H^\dagger = \sigma_z, \quad H \sigma_z H^\dagger = \sigma_x.
        \]
        Conclude that, on the bit-level, \((x, z)\) is mapped to \((z, x)\). (Ignore global phases for now.)
        
        \item[(b)] \textbf{Phase gate \(S\).} Recall that \(S = \begin{pmatrix}1 & 0 \\ 0 & i\end{pmatrix}\). Show that
        \[
        S \sigma_z S^\dagger = \sigma_z, \quad S \sigma_x S^\dagger = \sigma_y \quad (\text{up to a global phase } i).
        \]
        Conclude a rule in the bit form, for example \((x, z) \mapsto (x \oplus z, z)\). Verify this rule with a short calculation.
    \end{enumerate}
\end{question}

\begin{question}
    \textbf{Problem 4: CNOT on Two Qubits.}\\
    For a two-qubit system, let qubit \(c\) be the control and qubit \(t\) be the target. Denote the bit-string for the operator on qubit \(c\) by \((x_c, z_c)\) and on qubit \(t\) by \((x_t, z_t)\). We apply \(\text{CNOT}_{c \to t}\).
    \begin{enumerate}
        \item[(a)] Show explicitly how CNOT conjugates \(\sigma_x\) and \(\sigma_z\) on the control and target. (Hint: \(\text{CNOT}(\sigma_x \otimes I)\text{CNOT}^\dagger = ?\), etc.)
        
        \item[(b)] Based on part (a), deduce the update rule for \((x_c, z_c | x_t, z_t)\) in the bit-string representation. (You may give the final answer in terms of XOR operations.)
    \end{enumerate}
\end{question}

\section*{Stabilizer States and the Simulation Algorithm}
\textbf{Background (No Formal Group Theory Required).} An \(n\)-qubit \textit{stabilizer state} can be described as the unique +1 eigenstate of \(n\) independent, mutually commuting Pauli operators. In the \textit{binary matrix} approach, we represent those \(n\) operators by \(n\) rows in a \(2n\)-bit matrix (plus some extra bits for phases, if needed).

\begin{question}
    \textbf{Problem 5: Updating Stabilizer Generators Under Clifford Gates.}\\
    You now represent the quantum state using a list of \(n\) commuting \(n\)-qubit Pauli operators \(P_1, \ldots, P_n\), which stabilize the state:
    \[
    P_i \psi = \psi \quad \text{for all } i = 1, \ldots, n.
    \]
    These \(P_i\) are stored in bitwise form as \(n\) rows of a \(2n\)-bit matrix (each row corresponds to one operator).
    \begin{enumerate}
        \item[(a)] Suppose you apply a Clifford gate \(G\) (either \(H\), \(S\), or CNOT) to one or two qubits. Show that the new state \(G \psi\) is stabilized by the operators \(G P_i G^\dagger\). Give a short explanation of why this holds.
        
        \item[(b)] For each generator \(P_i\), apply the appropriate bitwise update rule (from Problems 3 and 4) to compute \(G P_i G^\dagger\) in bit-string form. Explain why the result is again a valid Pauli operator.
        
        \item[(c)] Conclude that, if each gate update requires only bitwise operations on each of \(n\) rows of \(2n\) bits, then the total simulation time for a circuit with \(T\) gates is polynomial in \(n\) and \(T\).
    \end{enumerate}
\end{question}

\begin{question}
    \textbf{Problem 6: Measurement in the Z Basis.}\\
    To \textit{measure} qubit \(k\) in the \(\sigma_z\) basis, we need to determine the outcome probabilities (0 vs. 1) and update the post-measurement state.
    \begin{enumerate}
        \item[(a)] If there is a stabilizer generator with a \(\sigma_z\) on qubit \(k\) (i.e., the bit pattern \((0, 1)\) at position \(k\)), then \(\psi\) is already an eigenstate of \(\sigma_z\) on that qubit. Show why the measurement outcome is \textit{deterministic} in that case.
        
        \item[(b)] If all stabilizer generators act on qubit \(k\) by \(\sigma_x\) or \(I\) (i.e., \((1, 0)\) or \((0, 0)\)), then the measurement outcome is random (50-50). Briefly explain how to \textit{update} the stabilizer matrix after obtaining a measurement result. (Hint: You may need to replace one generator with \(\sigma_z\) on that qubit, ensuring it still commutes with the others.)
    \end{enumerate}
\end{question}


\section*{Final Conclusion}
\begin{question}
    \textbf{Problem 7: Putting It All Together.}\\
    Using all the above steps, write a short argument (one paragraph) explaining why:\\
    
    \textit{Any circuit on \(n\) qubits using only \(\{\text{CNOT}, H, S\}\) gates, starting in the \(0^{\otimes n}\) state and ending in Z-basis measurements, can be simulated by a classical computer in time polynomial in \(n\) and in the number of gates.}\\
    
    You do not need to provide the constant factors; just argue that each gate and measurement can be handled by updating \(n\) bit-strings (the stabilizer generators), each of length \(2n\), in polynomial time.
\end{question}

\end{document}